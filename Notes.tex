\documentclass[12pt, a4paper]{article}

\usepackage{listings}
\usepackage{color}
\usepackage{enumitem}
\usepackage{amsthm}
\usepackage{amssymb}
\usepackage{listings}
\usepackage{setspace}
\usepackage{graphicx}
\graphicspath{ {./figures/} }


\title{General Physics: Mechanics, Heat, and Sound}
\author{William Darko}
\date{Fall 2021}

\pagenumbering{arabic}

\begin{document}

\maketitle
\newpage

\tableofcontents

\newpage

\section{Kinematics in 1D}
\paragraph*{}
Kinematics in 1 dimension refers to the motion of objects, on \textbf{1 spatial dimension},
not necessarily in a one dimensional space. One can still observe 1 dimensional
kinematics of objects in multidimensional spaces.

\subsection{Reference Frames}
\paragraph*{}
Measurements of \textbf{position, distance, speed} must be with respect to 
a \textbf{reference frame}.

\subsection{Displacement and Distance}

\subsubsection{Displacement}
\paragraph*{}
\textbf{Displacement} is the measure of \textbf{how far an object is from its initial (starting)
position, and its final postion}. Measure of displacement isn't dependant on the actual path
taken by the object, or the length of that path, all that matters is the starting position, and finishing position.

Displacement is defined as:

{
    \centering
    $\Delta x = x_2 - x_1$

}
where $x_2$ and $x_1$ are the final, and initial positions of the object, respectively. Displacement
is represented as a vector; a quantity with a \textbf{magnitude, and direction}.

\subsubsection{Distance}
\paragraph*{}
There's a difference between displacement, and distance. While displacement is 
only concerned with the absolute difference between the initial and final position of an object,
\textbf{distance} is concerned about the \textbf{absolute length of the path taken by the object}.
Distance is \textbf{represented as a scalar quantity}.


\subsection{Average Velocity, and Speed}

\subsubsection{Average Speed}
\paragraph*{}
\textbf{Defined as:}

{
    \centering
    \textbf{avg speed =} $\frac{\textrm{
        \textbf{\textit{distance travelled}}}}{\textrm{\textbf{\textit{time elapsed}}}}$ = $\frac{\textbf{d}}{\textbf{$\Delta t$}}$

}

\subsubsection{Average Velocity}
\paragraph*{}
Velocity is the change in rate of change in position, thus we're concerned with 
directional information; in what direction is our object moving. Thus average velocity 
is defined as:

{
    \centering
    \textbf{avg velocity =} $\frac{\textbf{\textit{displacement}}}{\textrm{\textbf{\textit{time elapsed}}}}$ = 
    {$\frac{\textbf{$\Delta x$}}{\textbf{$\Delta t$}}$}

}

\subsubsection{Instantaneous Velocity}
\paragraph*{}
Instantaneous velocity is the \textbf{average velocity as the time elapsed becomes infinitely small}.
In other words, as the limit of average velocity as time elapsed approaches 0:

{
    \centering
    \textbf{\textit{v} = } $\textrm{\textbf{\textit{lim}}}_{\textrm{$\Delta t \rightarrow 0$}}$ 
    \textbf{${\frac{\textrm{$\Delta x$}}{\textrm{$\Delta t$}}}$}

}

\subsection{Acceleration}
\paragraph*{}
Acceleration of an object is the \textbf{rate of change of that objects velocity}. Mathematically:

{
    \centering
    \textbf{avg acceleration =} $\frac{\textbf{\textit{ROC of velocity}}}{\textrm{\textbf{\textit{time elapsed}}}}$ = 
    \textbf{${\frac{\textrm{$\Delta v$}}{\textrm{$\Delta t$}}}$}

} 

\paragraph*{}
acceleration is also a vector quantity. In a single spatial dimension we really only care about
the sign. 

\textbf{Negative acceleration} is acceleration in the negative direction as define by the spatial coordinate system.

\textbf{Deceleration} occurs when acceleration is opposite in direction to velocity.

\textbf{Instantaneous acceleration} is the average acceleration as the time elapsed becomes infinitely small.
In other words, the limit of average velocity as the time interval approaches 0. 

{
    \centering
    \textbf{\textit{a } = } $\textrm{\textbf{\textit{lim}}}_{\textrm{$\Delta t \rightarrow 0$}}$ 
    \textbf{${\frac{\textrm{$\Delta v$}}{\textrm{$\Delta t$}}}$}

}

\end{document}